\section{통계적 가설검정}
한 연구는 효과적으로 가로등의 수를 늘림으로써 한 마을 안의 자동차사고를
줄일 수 있겠는가 하는 조사를 다루고 있다. 다음 표는 12곳에 가로등을 달기
1년 전과 1년 후의 매주 밤시간의 평균사고수이다. 이 자료는 가로등이 밤시간의
자동차사고를 줄였다고 할 수 있는 근거를 제공하는가?

\begin{tabularx}{0.9\textwidth}{ l|XXXXXXXXXXXX }
  \noalign{\smallskip}\hline\noalign{\smallskip}
  위치 & A & B & C & D & E & F & G & H & I & J & K & L \\
  전의 사고수 &
    8 & 12 & 5 & 4 & 6 & 3 & 4 & 3 & 2 & 6 & 6 & 9 \\
  후의 사고수 &
    5 &  3 & 2 & 1 & 4 & 2 & 2 & 4 & 3 & 5 & 4 & 3
\end{tabularx}

\subsection{풀이}

이 문제는 두 모평균에 대한 가설검정을 위해 대응표본을 사용하는 대응비교의 경우이다.
표본의 크기가 작기 때문에 표본정규분포를 사용할 수 없다. 따라서 $t$-분포를 대신 사용해야 한다.
R을 이용해 분석하면 아래와 같은 결과를 얻을 수 있다.

\VerbatimInput[numbers=none]{section-rout-4.txt}

`전의 사고수'가 `후의 사고수`보다 크다는 대립가설을 0.01보다 작은 신뢰수준에서도 기각되지 못하기 때문에,
`후의 사고수'는 `전의 사고수'보다 줄었다는 근거가 된다.

\subsection{스크립트}
R 스크립트는 아래와 같이 작성했다.

\VerbatimInput{section-4.R}
