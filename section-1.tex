\section{그래프에 의한 자료의 정리}

다음 자료는 어느 도시의 하루 중 최고기온을 8월 한 달간 기록한 것이다.
이 자료에 대하여 히스토그램, 점그림, 줄기-입그림을 그려 비교하라.

\begin{tabularx}{0.9\textwidth}{ l|X }
  \multicolumn{2}{r}{\small(단위: $^{\circ}\mathrm{C}$)} \\
  \noalign{\smallskip}\hline\noalign{\smallskip}
  도시 &
  29, 29, 34, 35, 35, 31, 32, 34, 38, 34, 33, 31, 31, 30, 34, 35 \\
  농촌 &
  34, 32, 32, 29, 28, 30, 29, 31, 29, 28, 30, 29, 29, 27, 28 \\
\end{tabularx}

도시의 최고기온을 분포를 히스토그램으로 그려보면
최고기온의 분포가 종 모양이고, 약간 좌쪽으로 치우친 일반형이다.
자료가 가운데로 집중되어 있다. 32도와 34도 사이의 최고기온을 보이는 경우가 가장 많고,
최고기온이 그보다 차이가 많을수록 경우가 더 줄어든다. 하지만 점그림을 확인해보면
최고기온이 36도와 38도 사이인 경우는 38도로 단 한번으로 특이점으로 볼 수 있다.
하지만 자료의 수가 충분히 많지 않기 때문에 특이점으로 단정하기는 쉽지 않다.

\subsection{도시의 분석}

\VerbatimInput[numbers=none]{section-img-1-1c.txt}

한편 줄기-입그림을 보더라도 30도에서 35도 사이가 가장 높은 빈도를 차지하는 것을
알 수 있다.

\begin{figure}[!hb]
  \begin{subfigure}{0.5\textwidth}
    \centering
    \includegraphics[width=\textwidth]{section-img-1-1a.pdf}
    \caption{도시의 히스토그램}
  \end{subfigure}
  \begin{subfigure}{0.5\textwidth}
    \centering
    \includegraphics[width=\textwidth]{section-img-1-1b.pdf}
    \caption{도시의 점그림}
  \end{subfigure}
  \caption{도시의 최고기온 그래프}
\end{figure}

\subsection{농촌의 분석}

농촌의 최고기온 분포를 히스토그램으로 그리면 좌측으로 치우친 그림이 나온다.
주로 27도와 29도 사이의 최고기온을 보인다. 그보다 높은 최고기온을 보이는 경우가 있지만,
온도가 높을수록 빈도는 줄어든다. 점그림을 보면 $y > -x$인 영역에는 자료가 없는
것을 알 수 있다. 즉, 27도와 34도 사이의 구간 내에서 최고기온이 높을수록 빈도는 낮게
분포되어 있다는 것을 알 수 있다.

\begin{figure}[!hb]
  \begin{subfigure}{0.5\textwidth}
    \centering
    \includegraphics[width=\textwidth]{section-img-1-2a.pdf}
    \caption{농촌의 히스토그램}
  \end{subfigure}
  \begin{subfigure}{0.5\textwidth}
    \centering
    \includegraphics[width=\textwidth]{section-img-1-2b.pdf}
    \caption{농촌의 점그림}
  \end{subfigure}
  \caption{농촌의 최고기온 그래프}
\end{figure}

\BVerbatimInput[numbers=none]{section-img-1-2c.txt}

줄기-입그림을 보면 최고기온이 32나 33도인 지점은 주변 구간에 비해 적은 것을 알 수
있는데, 통계량의 크기가 충분히 커지면 쌍봉우리형 분포를 보일 가능성이 있어보인다.

\subsection{도시와 농촌의 비교}

농촌의 최고기온 분포가 도시의 최고기온 분포보다 더 넓다. 도시의 최고기온 분포는
32도와 34도 사이를 중심으로 일반형의 분포도를 보이는데 반해, 농촌의 분포는
28도 29도의 최빈치 중심으로 왼쪽으로 치우친 분포를 보인다.

\subsection{스크립트}
그래프를 그리기 위해 아래와 같이 R 스크립트를 작성했다.

\VerbatimInput{section-1.R}

