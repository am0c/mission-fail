\section{실험계획법: 분산분석}
아래 표는 어떤 기계의 부품이 세 가지의 다른 윤활유를 사용할 때 마찰에
의해 손실된 무게이다. 윤활유의 종류에 따라 손실된 정도가 어떻게 다른가
유의수준 1\%로 검정하라. {\small(단위: mg)}

\begin{tabularx}{0.9\textwidth}{ l|XXXXXXXXXXX }
  \noalign{\smallskip}\hline\noalign{\smallskip}
  윤활유 X & 12 & 11 & 7 & 13 & 9 & 11 & 12 & 9 \\
  윤활유 Y & 8 & 10 & 7 & 5 & 6 & 10 & 7 & 8 & 11 & 7 & 8\\
  윤활유 Z & 9 & 3 & 7 & 8 & 4 & 6 & 6 & 5
\end{tabularx}

\subsection{검정}

반응결과에 영향을 주는 주요 원인을 찾기 위해 세 개 이상의 평균을 비교하여
분산분석하는 경우이다.

\VerbatimInput[numbers=none]{section-rout-5.txt}

\[
F = V_a / V_e = 11.035 > F(2, 24; 0.01) = 5.613
\]

기각역 $F(2, 24; 0.01)$은 R에서 \texttt{qf} 함수를 사용하여
\texttt{qf(0.99, 2, 24)}와 같이 구할 수 있다.

위 식을 계산하여 $F$-값이 기각역보다 더 크다는 것을 알 수 있다.
따라서 마찰에 의해 손실된 무게에는 유의한 차이가 있다고 할 수 있다.
유의수준 $0.01$에 비해 유의확률이 약 $0.0004$으로 아주 작은 것도 유의한 영향을 준다는 사실을 뒷받힘한다.
이에따라 이 요인의 모평균을 각 수준에서 비교할 필요가 있다.

\subsection{신뢰구간}

주어진 요인의 각 수준에서 모평균의 신뢰구간 범위를 구하려면 아래의 식을 사용한다.

\[
 \bar{x}_{i\cdot} \pm t\left(\phi_E; \frac{\alpha}{2}\right) \sqrt{\frac{V_E}{m}}
\]

또는 R에서 \texttt{confint} 함수를 사용하여 구간을 구할 수 있다.
그래프로 나타내면 \ref{ci}와 같다.

\begin{figure}[!hb]
  \centering
  \includegraphics[width=0.5\textwidth]{section-img-5-1.pdf}
  \caption{각 수준 모평균의 신뢰구간}
  \label{ci}
\end{figure}

\subsection{스크립트}
아래와 같이 R 스크립트를 작성했다.


\VerbatimInput{section-5.R}



