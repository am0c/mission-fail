\section{수치에 의한 자료의 정리}
다음은 어느 한 상장회사의 최근 25일간의 폐장 시 주가를 차례대로 기록한 것이다.

\begin{tabularx}{0.9\textwidth}{ l|X }
  \multicolumn{2}{r}{\small(단위: 원)} \\
  \noalign{\smallskip}\hline\noalign{\smallskip}
  국내기업 &
  13100, 13500, 12900, 12300, 13000, 13600, 13400, 14000, 14600, 15000,
  15700, 16300 \\
  국외기업 &
  15300, 15000, 14800, 15100, 15300, 15800, 16100, 16500, 16000, 15500,
  15900, 16000, 16000
\end{tabularx}

\begin{enumerate}
  \item 위 자료들에 대한 평균값, 중앙값, 최빈값을 구하여라.
  \item 위의 자료에 대한 범위, 표본분산, 표본표준편차, 변이계수 등을 구하라.
  \item 위의 값들에 대한 $25\%$ 백분위수 및 $75\%$ 백분위수를 구하라.
  \item 위의 주가자료에 대하여 상자그림을 그려라.
  \item 국내, 국외, 전체 기업에 대하여 신뢰구간을 구하라.
\end{enumerate}

R 스크립트를 작성하여 아래와 같이 결과를 얻었다.

\VerbatimInput[numbers=none]{section-rout-2.txt}

\subsection{기업의 분석}
\begin{figure}[!hb]
  \begin{subfigure}{0.5\textwidth}
    \centering
    \includegraphics[width=\textwidth]{section-img-2-1a.pdf}
    \caption{국내기업의 상자그림}
  \end{subfigure}
  \begin{subfigure}{0.5\textwidth}
    \centering
    \includegraphics[width=\textwidth]{section-img-2-2a.pdf}
    \caption{국외기업의 상자그림}
  \end{subfigure}
  \caption{기업의 상자그림}
\end{figure}

\subsection{국내기업}

평균 13950원이고 최소 12300원, 최고 16300원이다.
$25\%$ 백분위수는 13080원이고 $75\%$ 백분위수는 14700원이다.
범위와 표본분산, 표본표준편차, 변이계수는 위에 나타내었다.
상자그림을 보면 국내기업은 평균보다 낮은 주가를 중심으로 더 밀집되어 있다.

\subsection{국외기업}

평균 15640원이고 최소 14800원, 최고 16500원이다.
$25\%$ 백분위수는 15300원이고 $75\%$ 백분위수는 16000원이다.
범위와 표본분산, 표본표준편차, 변이계수는 위에 나타내었다.
상자그림을 보면 국외기업은 평균보다 높은 주가를 중심으로 더 밀집되어 있다.

\subsection{전체기업}

평균 14830원이고 최소 12300원, 최고 16500원이다.
$25\%$ 백분위수는 13600원이고 $75\%$ 백분위수는 15900원이다.
범위와 표본분산, 표본표준편차, 변이계수는 위에 나타내었다.
국내와 국외의 기업 상자그림을 비교하면 두 기업의 분포도는 비슷하지만,
외국기업이 전체적으로도 더 높은 평균 주가를 가지고 있고, 평균 주가보다
더 높은 주가를 중심으로 더 밀집되어 있다는 것을 알 수 있다.

%% TODO - Confidence Interval 분석

\subsection{스크립트}
아래와 같이 R 스크립트를 작성했다.

\VerbatimInput{section-2.R}
