\section{통계적 가설검정}
전산공장에서는 단위길이당 저항의 측면에서 두 가지 형태의 전선을 비교하려고
한다. 전선 1의 30개 표본과 전선 2의 35개 표본은 아래와 같은 결과를 냈다.

\begin{tabularx}{0.9\textwidth}{ l|X }
  \noalign{\smallskip}\hline\noalign{\smallskip}
  전선 1 & 
    55.2 53.5 52.3 54.1 52.4 50.5 53.5 46.9 52.9 57.1
    55.7 51.2 55.2 57.4 53.9 58.1 50.6 59.4 51.8 50.8
    56.9 56.3 59.1 52.7 56.1 58.2 53.1 50.6 53.1 59.7 \\
  전선 2 &
    46.9 50.6 47.3 48.0 49.2 48.4 48.5 48.6 48.2 50.2
    47.2 50.3 49.1 48.2 47.4 48.1 49.4 47.4 49.7 49.1
    49.3 50.3 50.8 48.3 47.7 48.5 51.1 50.9 49.5 49.7
    51.4 48.1 49.7 50.9 48.6
\end{tabularx}

이 결과들을 근거로 두 표본집단의 평균저항이 다르다고 말할 수 있겠는가?
{\small(유의수준=0.05)}

아래와 같은 검정 결과를 얻었다.

\VerbatimInput[numbers=none]{section-rout-3.txt}





