\section{통계적 가설검정}
전산공장에서는 단위길이당 저항의 측면에서 두 가지 형태의 전선을 비교하려고
한다. 전선 1의 30개 표본과 전선 2의 35개 표본은 아래와 같은 결과를 냈다.

\begin{tabularx}{0.9\textwidth}{ l|X }
  \noalign{\smallskip}\hline\noalign{\smallskip}
  전선 1 & 
    55.2 53.5 52.3 54.1 52.4 50.5 53.5 46.9 52.9 57.1
    55.7 51.2 55.2 57.4 53.9 58.1 50.6 59.4 51.8 50.8
    56.9 56.3 59.1 52.7 56.1 58.2 53.1 50.6 53.1 59.7 \\
  전선 2 &
    46.9 50.6 47.3 48.0 49.2 48.4 48.5 48.6 48.2 50.2
    47.2 50.3 49.1 48.2 47.4 48.1 49.4 47.4 49.7 49.1
    49.3 50.3 50.8 48.3 47.7 48.5 51.1 50.9 49.5 49.7
    51.4 48.1 49.7 50.9 48.6
\end{tabularx}

이 결과들을 근거로 두 표본집단의 평균저항이 다르다고 말할 수 있겠는가?
{\small(유의수준=0.05)}

아래와 같은 검정 결과를 얻었다.

\VerbatimInput[numbers=none]{section-rout-3.txt}

두 표본의 자료가 30개를 넘기기 때문에 두 모집단이 정규분포라고 가정할 수 있다.

\begin{align*}
  H_0: u_1 - u_2, = 0, H_1: u_1 - u_2 \not= 0
\end{align*}

두 모집단이 정규분포라고 말할 수 있으므로 $F$-검정을 하여 두 모분산도 같은지 추정한다.
$p$-값이 유의수준인 0.05보다 훨씬 작으므로 $H_1$은 기각되지 않는다.
즉, 두 모집단의 표본평균은 유의수준 $5\%$ 내에서 같다고 할 수 없다.
이를 베렌스-피셔 문제라고 한다.

따라서 Welch 또는 새터스웨이터(Satterthwate) 방법을 사용해야한다.
R에서는 \texttt{t.test} 함수의 인자에 \texttt{var.equal = FALSE}라고 인자를 지정해주어야 한다.
이렇게 $t$-검정을 한 결과를 보면 $p$-값이 $1.264e-10$으로 유의수준인 0.05보다 훨씬 작기 때문에,
`전선 1'의 모평균이 `전선 2'의 모평균과 같지 않다는 대립가설을 기각하지 못한다.

따라서 `전선 1'의 모평균과 `전선 2'의 모평균은 유의수준 0.05에서 다르다고 말할 수 있다.

\subsection{스크립트}
아래와 같이 R 스크립트를 작성했다.

\VerbatimInput{section-3.R}





